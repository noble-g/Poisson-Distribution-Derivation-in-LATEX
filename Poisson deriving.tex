\documentclass[a4paper]{article}

\usepackage[english]{babel}
\usepackage[utf8x]{inputenc}
\usepackage{amsmath} % Required for align* environment
\usepackage{graphicx}
\usepackage[colorinlistoftodos]{todonotes}
%\usepackage[margin=1.2in]{geometry}
\usepackage{geometry}
\geometry{margin=1in} % Adjust margins as needed

\title{Poisson Distribution Proving}
\author{Oloyede Abdulganiyu}
\date{March 07, 2024}

\begin{document}
\maketitle

\begin{abstract}
This is a LATEX project as well as a poisson distribution project, helping me practice the provings of poisson distributions I have learnt in my days as a statistics student of University of Ilorin.
\end{abstract}

\section{Problem}
\begin{center}
Let X~ Poisson(\lambda)\\
\text{where} \lambda > 0.\\
Find the variance of X
\end{center}

\section{Solution}
\begin{center}
Before we could solve for the variance of the poisson distribution, we first have to solve for the mean since the var(X) could be denoted by:\\
var(x) &= E(x^2) - \mu^2 \\
&= E(x^2) - [E(x)]^2\\
We are also going to note the series expansion of $\exp(a)$  as:\\
\exp(a) = \sum_{y=0}^{\infty}\frac{a^y}{y!}
\end{center}

\subsection{Solving for Mean}
\begin{center}
Using the Poisson probability mass function (PMF) formula:
\begin{align*}
f(x) = P(X = x) &= \frac{\lambda^x \exp(-\lambda)}{x!}\\
\text{and mean is } E(x) \text{ given by:}\\
E(x) &= \sum_{x=0}^{\infty} xf(x) \\
&= \sum_{x=0}^{\infty} x \frac{\lambda^x \exp(-\lambda)}{x!}
\end{align*}

We spread out the summation:

\begin{align*}
&= \frac{(1\lambda^1 \exp(-\lambda))}{1!} + \frac{(2\lambda^2 \exp(-\lambda))}{2!} + \frac{(3\lambda^3 \exp(-\lambda))}{3!} + \frac{(4\lambda^4 \exp(-\lambda))}{4!} + \dots + \\
&\quad \frac{(\infty\lambda^\infty \exp(-\lambda))}{\infty!}
\end{align*}

Simplifying the expression:

\begin{align*}
&= \lambda \exp(-\lambda) + \lambda^2 \exp(-\lambda) + \lambda^3 \exp(-\lambda) + \dots + \lambda^\infty \exp(-\lambda) \\
&= \lambda \exp(-\lambda) (1 + \lambda + \frac{\lambda^2}{2!} + \frac{\lambda^3}{3!} + \frac{\lambda^4}{4!} + \dots) \\
%\text{using the exponential series expansion stated earlier, the summation series above will be equal to} \exp(\lambda)\\
&= \lambda \exp(-\lambda) \exp(\lambda) \\
&= \lambda \exp(-\lambda+\lambda)  \\
&= \lambda
\end{align*}

\end{center}

\subsection{Solving for Variance}
\begin{center}
Using the Poisson probability mass function (PMF) formula stated in section 2 above:
\begin{align*}
\text{Var}(X) &= E((X - \mu)^2) \\
&= E(X^2) - \mu^2\\
\text{And as we've solved in the equation of  mean above, } \mu = \lambda \text{, thus:}\\
&= E(X^2) - \lambda^2
\end{align*}

We need to calculate \(E(X^2)\) first. 
\begin{align*}
E(X^2) &= \sum_{x=0}^{\infty} x^2 f(x) \\
&= \sum_{x=0}^{\infty} x^2 \frac{\lambda^x \exp(-\lambda)}{x!}\\
\text{you'll agree with me} \\
x^2 = x (x-1) + x \\
\thus \\
E(x^2) &= E(x (x-1) + x) \\
&= E(x (x-1)) + E(x)\\
\text{and as we've already established} E(x) = \lambda\\
\text{so we'll focus on the first part: } E(x (x-1))\\
E(x (x-1)) &=  \sum_{x=0}^{\infty} x(x-1) \frac{\lambda^x \exp(-\lambda)}{x!}\\
&= \exp(-\lambda) \lambda^2 \sum_{x=2}^{\infty} x(x-1) \frac{\lambda^{x-2} }{(x-2)!}\\
\text{let} y = x-2 \text{so,}\\
&= \exp(-\lambda) \lambda^2 \sum_{y=0}^{\infty} x(x-1) \frac{\lambda^{y} }{y!}\\
%&= \exp(-\lambda) \sum_{x=0}^{\infty} x^2 \frac{\lambda^x }{x!}\\
%&= \exp(-\lambda) \lambda^2 \sum_{x=2}^{\infty} x(x-1)\frac{\lambda^{x-2} }{x(x-1)(x-2)!}\\
%&= \exp(-\lambda) \lambda^2 \sum_{x=2}^{\infty}\frac{\lambda^{x-2}}{(x-2)!}\\
%\text{let x-2 be y}\\
%&= \exp(-\lambda) \lambda^2 \sum_{y=0}^{\infty}\frac{\lambda^{y}}{(y)!}\\
\text{and once again we have our exponential expansion series}\\
&= \exp(-\lambda) \lambda^2 \exp(\lambda) \\
&= \lambda^2 \exp(-\lambda +\lambda )\\
%&= \lambda^2 \exp(-\lambda +\lambda)\\
&= \lambda^2 \exp(0)\\
E(x(x-1) &= \lambda^2\\
\text{then bringing it back up to} x^2\\
E(x^2) = E(x(x-1) + x) = E(x(x-1)) + E(x) = \lamda^2 + \lambda \\
\text{and back to variance}\\
\text{substituting this back into the variance formula:}\\
\text{Var}(X) &= E(X^2) - \lambda^2\\
&= \lamda^2 + \lambda - (\lambda)^2 \\
&= \lambda
\end{align*}

Finally, substituting this back into the variance formula:\\
So, the variance of a Poisson distribution with parameter \( \lambda \) is \(\lamda\) too.
\end{center}

\section{ References:}
$
[1] 
https://proofwiki.org/wiki/Power_Series_Expansion_for_Exponential_Function
\newline
[2] http://www.math.com/tables/expansion/exp.htm
\newline
[3] https://en.wikipedia.org/wiki/Discrete_cosine_transform.
\newline
[4] ...
\newline
[5] ... .
\newline
[6] https://www.overleaf.com .
$
\end{document}